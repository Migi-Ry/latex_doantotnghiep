\section{Thực trạng}
Hiện nay, ở nước ta đang xuất hiện một thực trạng đáng buồn là tình trạng sinh viên chọn sai ngành nghề tại Việt Nam đang ở mức báo động với tỷ lệ lên đến 60\%, theo khảo sát của Trung tâm Dự báo nhu cầu nhân lực và thông tin thị trường lao động năm 2019 \cite{tienphong}. Con số này vẽ nên một bức tranh ảm đạm về thị trường lao động, lãng phí nguồn lực quốc gia và tiềm ẩn nhiều hệ lụy nghiêm trọng. Không chỉ vậy, trong năm 2023, theo kết quả của Viện nghiên cứu Đào tạo Kinh tế Quốc tế tại thành phố Hồ Chí Minh cho thấy rằng 15-20\% sinh viên đã tốt nghiệp mới nhận ra mình đã chọn sai ngành, nghề học \cite{thanhnien1}, điều này đã và đang góp phần dẫn đến một hệ lụy vô cùng nghiêm trọng, đó là hằng năm khoảng hơn 400.000 sinh viên ra trường nhưng trong đó có khoảng gần 200.000 lao động thất nghiệp, tổng quan hơn cả nước có 1,3 triệu người thất nghiệp khi trong độ tuổi lao động (theo Tổng cục Thống kê 2021) \cite{thanhnien2}. 

Điều này đã ảnh hưởng vô cùng tiêu cực tới tình hình kinh tế, xã hội. Đầu tiên là tăng gánh nặng lên hệ thống an sinh xã hội. Khi tỉ lệ thất nghiệp tăng, nhu cầu về các dịch vụ an sinh xã hội như trợ cấp thất nghiệp, chăm sóc sức khỏe, và hỗ trợ giáo dục cũng tăng lên, từ đó tạo ra một gánh nặng khổng lồ lên hệ thống an sinh xã hội. Thứ hai, việc thất nghiệp gây nên một tâm lý tự ti, tiêu cực dẫn đến căng thẳng và tuyệt vọng, điều này góp phần đẩy mạnh tình trạng trộm cắp và bạo lực. Thứ ba, việc số lượng người thất nghiệp lớn ảnh hưởng tới thị trường lao động và việc làm, giảm năng suất lao động, gây ra sự giảm sút về sản phẩm cũng như GDP của quốc gia, hơn nữa việc người thất nghiệp thường có chi tiêu thấp hơn do sự thiếu hụt về tài chính cũng ảnh hưởng lớn đến thị trường phân phối và tiêu thụ sản phẩm, bào mòn tốc độ phát triển kinh tế. Một số ví dụ tiêu biểu mà ta có thể bắt gặp trên thế giới hiện nay là Nam Phi, quốc gia có tỉ lệ thất nghiệp lớn nhất thế giới, dù được coi là quốc gia phát triển, giàu có và hùng mạnh bậc nhất Châu Phi nhưng lại có tỉ lệ thất nghiệp 32,4\% vào năm 2023 \cite{Investopedia}. Điều này cũng gây ra một thực trạng kinh tế đáng buồn khi trong năm 2023 quốc gia này chỉ tăng trưởng 0.6\% GDP, chịu tình trạng thiếu điện kéo dài tới 289 ngày trong cùng năm và giá cả tăng phi mã cũng ảnh hưởng to lớn tới những người có thu nhập thấp. Gần hơn và quen thuộc hơn đối với đất nước ta đó là Nhật Bản. Trong năm 2023, đất nước này đã trải qua một năm có tình trạng kinh tế vô cùng tồi tệ khi GDP chỉ tăng trưởng khoảng 1.8\%, lạm phát tiêu dùng tăng 2,8\% so với cùng kỳ năm trước dẫn đến hàng hóa vô cùng đắt đỏ, tồi tệ nhất có lẽ là việc đồng Yên Nhật sụt giá thảm hại (trượt giá 11,5\% so với đồng đô la Singapore). Những điều trên chịu tác động không nhỏ từ việc già hóa dân số và thiếu hụt lực lượng lao động của đất nước này. Ngược lại, quốc gia có tỉ lệ thất nghiệp thấp nhất thế giới hiện tại là Qatar với khoảng 0,8\% dân số thất nghiệp, có mức thu nhập bình quân đầu người khoảng hơn 80 nghìn đô la Mỹ, và vừa chi hơn 200 tỷ đô la để tổ chức sự kiện bóng đá hoành tráng bậc nhất toàn cầu là World Cup 2022 \cite{Investopedia}.

Đối với nước ta - một đất nước đang bước vào thời kỳ dân số vàng, để tận dụng được tốt thời cơ này nhằm đưa nước ta sánh vai với các cường quốc năm châu thì việc giảm tỉ lệ thất nghiệp và việc hỗ trợ lựa chọn ngành học là một trong những chủ trương hàng đầu của Đảng và nhà nước, được thúc đẩy qua nhiều chương trình, chính sách hỗ trợ như chính sách về bảo hiểm thất nghiệp hay chương trình phục hồi và phát triển thị trường lao động, v.v. Đối với Trường \acrshort{dhbk}, \acrshort{pgsts} Bùi Hoài Thắng - trưởng phòng đào tạo Trường \acrshort{dhbk} - cho rằng số sinh viên bỏ học, bị buộc thôi học hằng năm khá nhiều \cite{tuoitre1}, trong số này có không ít sinh viên không thích ngành học. Lý giải về việc sinh viên chọn sai ngành, thầy Thắng cho rằng tỉ lệ cha mẹ ép con chọn ngành theo ý mình vẫn còn rất nhiều. Điều này khiến xác suất thí sinh phải chọn ngành không theo sở trường và mong muốn của mình khá cao. ``Chọn ngành không đúng sở trường, việc học sẽ rất khó khăn. Có những bạn cố gắng có thể hoàn thành chương trình học nhưng cũng có người bỏ giữa chừng để chuyển ngành khác. Những người vượt qua được nhưng khi ra trường, đi làm khó phát huy năng lực của mình trong nghề" - thầy Thắng cũng chia sẻ thêm về vấn đề này. Đây là thực trạng đáng buồn và chúng em hy vọng đồ án này có thể góp phần nào sức lực để cải thiện tình trạng này. 
