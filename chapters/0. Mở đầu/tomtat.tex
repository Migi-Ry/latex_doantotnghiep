Nghiên cứu và phát triển Hệ thống hỗ trợ quyết định (Decision Support System - \acrshort{dss}) bao gồm hệ thống website lấy ý tưởng từ Ikigai - Lý do để sống của người Nhật nhằm hỗ trợ học sinh lựa chọn chuyên ngành học thuật, ngành nghề tương lai. Mục tiêu chính của hệ thống là giúp học sinh khám phá ra niềm đam mê, sở trường của mình và cách chúng phù hợp với nhu cầu của xã hội, từ đó thúc đẩy các bạn học sinh đưa ra những quyết định hợp lý về con đường học vấn và nghề nghiệp.

Phương pháp được thực hiện trên hệ thống bao gồm việc mô hình hóa bài toán sử dụng sơ đồ Ikigai Venn với 4 câu hỏi: ``Bạn yêu thích điều gì?", ``Bạn giỏi gì?", ``Bạn có thể được trả tiền để làm gì?" và ``Thế giới cần gì?". Việc giải quyết được thực hiện bằng cách trả lời cho từng câu hỏi, sau đó tổng hợp và tính toán kết quả bằng các phương pháp giải quyết vấn đề phù hợp dựa trên Phân tích Quyết định Đa tiêu chí (Multi-Criteria Decision Making - \acrshort{mcdm}). Phương pháp \acrfull{mbti} được sử dụng để tìm hiểu tính cách của người dùng trả lời cho câu hỏi ``Bạn yêu thích điều gì?", trong khi Nhóm ngành nghề (Career Clustering) giúp người dùng xác đinh nhóm ngành dựa trên thế mạnh trả lời cho câu hỏi ``Bạn giỏi gì?". Dữ liệu từ nhiều tổ chức khác nhau như tổng cục thống kê, phòng thống kê của tổ chức lao động thế giới ILOSTAT và nhiều nguồn uy tín khác được tổng hợp để trả lời các câu hỏi liên quan đến ``Bạn có thể được trả tiền để làm gì?" và ``Thế giới cần gì?". Các giải thuật về Weight-sum và Vikor trong bài toán phân tích quyết định đa tiêu chí được sử dung để tính toán giải pháp tốt nhất.

Dưới sự gợi ý của hệ thống hỗ trợ quyết định (\acrshort{dss}), học sinh sẽ có được hiểu biết sâu sắc hơn về đam mê, sở trường và các con đường nghề nghiệp tiềm năng, từ đó dẫn đến sự tự tin và rõ ràng hơn trong quá trình ra quyết định. Khảo sát sơ bộ trên 100 sinh viên mang lại kết quả tích cực, cho thấy sự hài lòng của học sinh về hệ thống, cũng như độ chính xác ở mức tốt. Những phát hiện này cho thấy hiệu quả của phương pháp được triển khai và mở đường trong tương lai cho việc đánh giá và tinh chỉnh thêm để liên tục nâng cao khả năng của hệ thống trong việc hỗ trợ học sinh trên hành trình học tập và nghề nghiệp.