\section{Yêu cầu phi chức năng (Non-functional Requirements}

Để đảm bảo hiệu suất và trải nghiệm người dùng tốt nhất, hệ thống phải đáp ứng một số yêu cầu không chỉ về chức năng mà còn về cách thức hoạt động và hiệu suất. Dưới đây là các yêu cầu không chức năng của hệ thống:
\begin{itemize}
    \item \textit{Đa Nền Tảng và Đa Trình Duyệt:} Hệ thống cần hoạt động ổn định trên nhiều nền tảng và trình duyệt khác nhau như Chrome, Firefox để đảm bảo sự linh hoạt và tiện lợi cho người dùng.
    \item \textit{Giao Diện Thân Thiện:} Giao diện người dùng phải được thiết kế sao cho thân thiện, dễ hiểu và dễ sử dụng để người dùng có thể nhập thông tin một cách dễ dàng và thuận tiện.
    \item \textit{Tùy Chỉnh và Điều Chỉnh:} Hệ thống cần có khả năng tùy chỉnh và điều chỉnh để phù hợp với sở thích, kỹ năng và giá trị cá nhân của từng người dùng.
    \item \textit{Phản Hồi và Cập Nhật Thường Xuyên:} Hệ thống cần cung cấp phản hồi và cập nhật thường xuyên dựa trên dữ liệu mới và phản hồi từ người dùng để cải thiện chất lượng của quá trình tư vấn.
    \item \textit{Bảo Mật và Bảo Vệ Thông Tin Cá Nhân:} Hệ thống phải đảm bảo bảo mật và bảo vệ thông tin cá nhân của người dùng một cách tốt nhất có thể để tránh rủi ro về việc lộ thông tin.
    \item \textit{Minh Bạch và Nguồn Gốc Dữ Liệu:} Hệ thống cần minh bạch về cách thức hoạt động và nguồn gốc của dữ liệu và thông tin được sử dụng để tư vấn người dùng, giúp tăng cảm giác tin cậy từ phía người dùng.
    \item \textit{Thời Gian Chờ Phản Hồi:} Hệ thống cần hoạt động ổn định, thời gian chờ phản hồi từ hệ thống không nên vượt quá 10 giây để đảm bảo trải nghiệm người dùng mượt mà và không bị gián đoạn.
\end{itemize}