\section{Kết quả}
    Qua quá trình thực hiện đồ án, nhóm nghiên cứu đã thu hoạch được nhiều bài học vô cùng quý giá và bổ ích. Không chỉ được nghiên cứu về kiến thức chuyên môn về bài toán quyết định đa tiêu chí (MCDM) và mô hình Ikigai, mà còn học hỏi được những kỹ năng mềm thiết yếu trong quá trình nghiên cứu khoa học và thực hiện dự án.
    
    Đầu tiên, các thành viên nhóm được rèn luyện về phương pháp nghiên cứu khoa học. Từ bước quan sát thực tế, đặt ra những vấn đề cần giải quyết, tìm kiếm phương pháp tiếp cận, xây dựng mô hình và hệ thống giải quyết vấn đề, cho đến thực hiện và đánh giá kết quả, các thành viên trong nhóm đã trải nghiệm đầy đủ các bước trong một quá trình nghiên cứu bài bản. Nhờ vậy, nhóm hiểu rõ tầm quan trọng của việc xác định rõ mục tiêu, lập kế hoạch chi tiết, sử dụng các nguồn tài liệu khoa học uy tín, và áp dụng các phương pháp nghiên cứu phù hợp để đạt được kết quả chính xác và khách quan.
    
    Bên cạnh đó, thành viên tham gia nghiên cứu đề tài cũng được rèn luyện kỹ năng làm việc nhóm hiệu quả. Việc phân chia công việc hợp lý, giao tiếp cởi mở và phối hợp nhịp nhàng giữa các thành viên là yếu tố then chốt để hoàn thành dự án đúng tiến độ và chất lượng. Tất cả thành viên đã học được cách lắng nghe ý kiến, tôn trọng sự khác biệt, và hỗ trợ lẫn nhau để cùng nhau vượt qua những khó khăn và thử thách trong quá trình thực hiện.
    
    Đặc biệt, sinh viên tham gia xây dựng đề tài còn được rèn luyện kỹ năng quản lý dự án. Từ việc lên kế hoạch chi tiết, phân bổ nguồn lực hợp lý, theo dõi tiến độ thực hiện, cho đến xử lý các tình huống bất ngờ như khi có thành viên bỏ không hoàn thành, thành viên không theo tiến độ,... và đánh giá rủi ro, thành viên nhóm đã học được cách điều phối dự án một cách hiệu quả. Kỹ năng này giúp thành viên nhóm tự tin hơn trong việc đảm nhận những vai trò lãnh đạo và tổ chức trong tương lai.
    
    Hơn cả những kiến thức và kỹ năng, đồ án còn giúp thành viên nhóm phát triển tinh thần trách nhiệm và ý thức học tập. Mỗi thành viên hiểu rằng để thành công, cần phải nỗ lực hết sức, hoàn thành tốt nhiệm vụ được giao và luôn sẵn sàng học hỏi những điều mới.
    Có thể nói, đồ án chuyên ngành cũng như đồ án tốt nghiệp là một trải nghiệm vô cùng quý giá đối với mỗi thành viên trong nhóm nghiên cứu nói riêng cũng như tất cả sinh viên tại đại học Bách Khoa nói chung. Những bài học thu được từ quá trình thực hiện sẽ là hành trang hữu ích cho từng thành viên trong việc học tập và phát triển sau này.
    
